% POMACS (journal-style) submission template for ACM SIGMETRICS
% - Single column (acmart acmsmall)
% - Review + anonymous for double-blind submission
% - Page numbers enabled for review
% Switch to camera-ready per instructions in README.md

\documentclass[acmsmall,screen,review,anonymous]{acmart}

% Journal metadata (filled at camera-ready; omit for submission)
\acmJournal[POMACS]{POMACS}
% \\acmVolume{X}\\acmNumber{Y}\\acmArticle{ZZ}\\acmYear{2026}\\acmMonth{6}

% Review housekeeping
\settopmatter{printacmref=false} % hide reference format block during review
\renewcommand\footnotetextcopyrightpermission[1]{} % suppress permission footnote
\pagestyle{plain} % page numbers

% Packages: keep minimal and acmart-compatible
\usepackage{microtype}
\usepackage{booktabs}   % tables
\usepackage{siunitx}    % units, alignment
\usepackage{subcaption} % subfigures
\usepackage[capitalize,noabbrev]{cleveref}
\usepackage{tikz}       % placeholder figure
% Shared macros for statistics, references, and units.
% Load siunitx in the main document before including this file.

% Confidence interval formatting: \ci{mean}{low--high} -> "123 (95\% CI: 120--126)"
\newcommand{\ci}[2]{#1\,\,(95\% CI: #2)}

% Percentage helper: \perc{12.3} -> "12.3\%"
\newcommand{\perc}[1]{#1\%}

% Common reference helpers
\newcommand{\figref}[1]{Fig.~\ref{#1}}
\newcommand{\tabref}[1]{Table~\ref{#1}}
\newcommand{\secref}[1]{\S\ref{#1}}
\newcommand{\thmref}[1]{Theorem~\ref{#1}}

% Units helper (requires siunitx): "\unit{ms}" -> typeset unit with spacing
\newcommand{\unit}[1]{\,\si{#1}}

% Inline code style
\newcommand{\code}[1]{\texttt{#1}}

% Theorem environments (customize as needed)
\theoremstyle{plain}
\newtheorem{theorem}{Theorem}
\newtheorem{lemma}{Lemma}
\theoremstyle{definition}
\newtheorem{definition}{Definition}
\theoremstyle{remark}
\newtheorem*{remark}{Remark}

% Optional macros (keep conservative under acmart)
\newcommand{\etal}{et~al.}
\sisetup{detect-all=true}

% Simple placeholder figure macro (compiles without external files)
\newcommand{\placeholderfigure}[1][Placeholder figure; replace with your result]{%
  \begin{tikzpicture}
    \draw[gray] (0,0) rectangle (8,4);
    \node[align=center] at (4,2) {#1};
  \end{tikzpicture}%
}

\begin{document}

\title[Short Title]{Full Title of Your Paper: Specific Method + Measurable Outcome}

% Anonymous for review; uncomment and fill after acceptance
% \\author{First A. Author}
% \\affiliation{% 
%   \\institution{University/Lab}
%   \\city{City}
%   \\country{Country}}
% \\email{first.last@example.edu}

% \\author{Second B. Author}
% \\affiliation{% 
%   \\organization{Organization}
%   \\city{City}
%   \\country{Country}}
% \\email{second@example.org}

\begin{abstract}
Context/problem (1 sentence). Why existing approaches fall short (1 sentence).
Your key idea and what is novel (1--2 sentences). Headline quantitative
results with uncertainty and scope (1--2 sentences). Implications (1 sentence).
\end{abstract}

% CCS Concepts and Keywords (can be added or refined at camera-ready)
% Use the ACM tool to generate CCSXML: https://dl.acm.org/ccs
% \\begin{CCSXML}
% ...
% \\end{CCSXML}
% \\ccsdesc[500]{Theory of computation~Queueing theory}
% \\keywords{measurement, performance evaluation, scheduling, reproducibility}

\maketitle
\setcounter{footnote}{0} % ensure permission footnote counter is reset

\section{Introduction}
% Motivate with a concrete pain point; explain what's hard.
% Bullet your contributions; each should map to a section/result.
% Preview one main figure or theorem with intuition.

\begin{itemize}
  \item C1: <Method/Theorem> achieving <capability>, validated by <key result>.
  \item C2: <System/Prototype> with <feature>; evaluated on <workloads>.
  \item C3: <Measurement insight> explaining <phenomenon>.
\end{itemize}

\section{Background and Model}
% Scope, assumptions, notation. For systems, architecture/workloads.
% For theory, model definitions and operational semantics.
% Be explicit about limits and what is out of scope.

\section{Methodology}
% Theory: proof roadmap, lemmas, intuition; tightness/bounds.
% Empirical/measurement: datasets, instrumentation, metrics, baselines,
% setup details (HW/SW), seeds, repetitions, CIs, tests.
% Simulation: model fidelity, warmup, convergence, validation.

\section{Results}
% Organize around RQs or hypotheses. Include uncertainty (CIs), baselines,
% sensitivity/ablations, and practical significance (not only statistical).
% Example figure and table below.

\begin{figure}[t]
  \centering
  % Placeholder drawn with TikZ, no external file required
  \placeholderfigure[Headline result goes here]
  \caption{Headline result: succinct, outcome-focused caption.}
  \Description{Textual description of the figure for accessibility.}
  \label{fig:headline}
\end{figure}

\begin{table}[t]
  \caption{Performance summary across workloads (95\% CIs).}
  \label{tab:summary}
  \centering
  \begin{tabular}{l S[table-format=2.1] S[table-format=2.1] S[table-format=2.1]}
    \toprule
    {Workload} & {Baseline} & {Method} & {Delta [\%]} \\
    \midrule
    A &  45.2 &  36.0 & -20.4 \\
    B & 120.5 &  95.9 & -20.4 \\
    C &  78.3 &  60.2 & -23.2 \\
    \bottomrule
  \end{tabular}
\end{table}

\section{Threats to Validity and Ethics}
% Internal/external/construct validity; limitations; IRB/ethics if applicable.

\section{Related Work}
% Position precisely relative to prior theory/systems/measurement literature.

\section{Conclusion}
% Key takeaways, limitations, and implications for practice/research.

% \\begin{acks}
% Add acknowledgments after acceptance; remove anonymous/review options,
% restore \\settopmatter{printacmref=true}, and include funding details.
% \\end{acks}

\bibliographystyle{ACM-Reference-Format}
\bibliography{references}

\appendix

\section{Proofs and Additional Experiments}
% Detailed proofs, extended tables/figures, extra sensitivity analyses.

\end{document}